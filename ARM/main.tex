\def\mytitle{IMPLEMENTATION OF BOOLEAN LOGIC IN ARM}
\def\myauthor{PRATHYUSHA K}
\def\contact{reddyprathyushakorepu95@gmail.com}
\def\mymodule{Future Wireless Communication (FWC)}
\documentclass[10pt, a4paper]{article}
\usepackage[a4paper,outer=1.5cm,inner=1.5cm,top=1.75cm,bottom=1.5cm]{geometry}
\twocolumn
\usepackage{graphicx}
\graphicspath{{./images/}}
\usepackage[colorlinks,linkcolor={black},citecolor={blue!80!black},urlcolor={blue!80!black}]{hyperref}
\usepackage[parfill]{parskip}
\usepackage{lmodern}
\usepackage{tikz}
%\documentclass[tikz, border=2mm]{standalone}
\usepackage{karnaugh-map}
%\documentclass{article}
\usepackage{tabularx}
\usepackage{circuitikz}
\usetikzlibrary{calc}
\usepackage{enumitem}

\renewcommand*\familydefault{\sfdefault}
\usepackage{watermark}
\usepackage{lipsum}
\usepackage{xcolor}
\usepackage{listings}
\usepackage{float}
\usepackage{titlesec}

\titlespacing{\subsection}{1pt}{\parskip}{3pt}
\titlespacing{\subsubsection}{0pt}{\parskip}{-\parskip}
\titlespacing{\paragraph}{0pt}{\parskip}{\parskip}
\newcommand{\figuremacro}[5]{
   % \begin{figure}[#1]
      %  \centering
       % \includegraphics[width=#5\columnwidth]{#2}
        %\caption[#3]{\textbf{#3}#4}
        %\label{fig:#2}
    %\end{figure}
}

\lstset{
frame=single, 
breaklines=true,
columns=fullflexible
}

%\thiswatermark{\centering \put(181,-119.0){\includegraphics[scale=0.13]{iith_logo3}} }
\title{\mytitle}
\author{\myauthor\hspace{1em}\\\contact\\FWC22047\hspace{6.5em}IITH\hspace{0.5em}\mymodule\hspace{6em}ASSIGNMENT-8}
\begin{document}
	\maketitle
%	\tableofcontents
	\begin{abstract}
	    This manual shows how to implement XOR using NOR gates. \\
 
% \centering
 % \begin{center}
% \begin{tabular}{ |c |c |c |c |}

 %\hline
 %U  &  V  &  W  &  \hspace{3mm}G\\
 %\hline
 %0  &  0  &  0  &  \hspace{3mm}1\\
 %\hline
 %0  &  0  &  1  &  \hspace{3mm}0\\
 %\hline
 %0  &  1  &  0  &  \hspace{3mm}1\\
 %\hline
 %0  &  1  &  1  & \hspace{3mm}0\\
 %\hline
 %1  &  0  &  0  & \hspace{3mm}1\\
 %\hline
 %1  &  0  &  1  & \hspace{3mm}0\\
 %\hline
 %1  &  1  &  0  & \hspace{3mm}0\\
 %\hline
 %1  &  1  &  1  &  \hspace{3mm}1\\
 %\hline
 %\end{tabular}
 %\end{center}
	  	\end{abstract}
	

	\section{Components}
  \begin{tabularx}{0.48\textwidth} { 
  | >{\centering\arraybackslash}X 
  | >{\centering\arraybackslash}X 
  | >{\centering\arraybackslash}X | }
\hline
 \textbf{Components}& \textbf{Values} & \textbf{Quantity}\\
\hline
Vaman Board &  & 1 \\  
\hline
JumperWires& M-F & 5 \\ 
\hline
Breadboard &  & 1 \\
\hline
USB-C cable&  & 1 \\
\hline
\end{tabularx}

   \section{Implementation}

%\begin{center}
 %    \begin{karnaugh-map}[4][2][1][$YZ$][$X$]
  %      \minterms{0,2,4,7}
   %     \maxterms{1,3,5,6}
    %    \implicant{1}{5}
     %   \implicant{1}{3}
      %  \implicant{6}{6}
    %\end{karnaugh-map}
%\end{center}
%\begin{center}
%K-map
%\end{center}
 %   \paragraph{Karnugh Map :}
%The  minimized expression using the K-map can be expressed as
%\begin{equation}
%G=(V+W')(U+W')(U'+V'+W) 
%\end{equation}
{XOR GATE :-}
The output is HIGH(1) if only if one of the inputs is HIGH.If both the inputs are LOW or HIGH, then the output is LOW(0).When the two inputs are different it produce HIGH value.
\vspace{5MM}

\textbf{TRUTH TABLE :-}
\vspace{5MM}
\newline

\begin{tabular}{|c|c|c|}
\hline
\textbf{A} & {B} & {Q} \\
\hline
0 & 0 & 0 \\
\hline
0 & 1 & 1 \\
\hline
1 & 0 & 1 \\
\hline
1 & 1 & 0 \\
\hline


\end{tabular}
\vspace{3mm}
\\
\centering{\textbf{Table-1}}
\vspace{5MM}
\newline

\textbf{BOOLEAN EXPRESSION :-}
\vspace{5MM}
\newline
\begin{equation}
    Q = AB' + BA'
\end{equation}
\textbf{2. XOR with NOR Gates :-}


    $Q = AB' + BA'$

    $Q = A'.B + A.B' + A.A'+ B.B'$  

    $Q = A'(A + B) + B'(A + B)$

    $Q = (A + B) (A'+ B')$  (2)

Take complement on both sides to equation (2)

$Q' = ((A + B)(A' + B'))'$

$Q' = (A + B)' + (A'+ B')'$  (Demorgan's theorem)     (3)

Take complement on both sides to equation (3)

$Q = ((A + B)' + (A' + B')')'$


\textbf{CIRCUIT DIAGRAM}
\\
\vspace{3mm}



\begin{circuitikz} \draw

(0,2) node[nor port] (nor1) {}
(nor1.out) node(x0) [anchor=south west] {$x0=\overline{A}$}
(nor1.in 1) -- ++ (-2,0) node[circ]{} node[left]{$A$}

(1,0) node[nor port] (nor2) {}
(nor2.in 1) -- ++ (0,0) node[circ]{} node[left]{$B$}
(nor2.out) node(x1) [anchor=north west] {$x1=\overline{B}$}
(3,1) node[nor port] (nor3) {}
(nor3.out) node(x2) [anchor=south west] {$x2=\overline{x0+x1}$}
(3,-1) node[nor port] (nor4) {}
%(nor4.in 2) -- ++ (-4,0) node[circ]{} node[left]{$A$}
%(nor4.in 1) -- ++ (-2,0) node[circ]{} node[left]{$B$}
(nor4.out) node(x3) [anchor=north west] {$x3=\overline{A+B}$}
(5,0) node [nor port] (nor5) {}
(nor5.out) node(Q) [anchor=north west] {$Q=\overline{x2+x3}$}
(nor1.out) -| (nor3.in 1)
(nor2.out) -| (nor3.in 2)
(nor3.out) -| (nor5.in 1)
(nor4.out) -| (nor5.in 2)
(nor1.in 1) |- (nor4.in 2)
(nor2.in 1) |- (nor4.in 1);

\end{circuitikz}
\vspace{5MM}
\newline
\textbf{Fig-1}
\vspace{5MM}

    $ X0=A' $
    
    $ X1=B' $
    
    $ X2=(X0+X1)' $
    
    $ X3=(A+B)' $

        $ Q=(X2+X3)' $
    
Download the following code and execute


The code below realizes the Boolean logic XOR gate using nor gates using Vaman Board
\\
2,4 GPIO Pins of Vaman Board are configured as input pins and the required Logic for A and B are drawn from 5V (Digital '1'),GND (Digital '0'). Built in led at 22nd pin will glow based on Q satisfying the Truth table.
\begin{center}
\fbox{\parbox{8.5cm}{\url{https://github.com/Prathyushakorepu/FWC/blob/main/arm/codes/src/main.c}}}
\end{center}
\section{Setup}
\begin{enumerate}
\item Connect the Vaman to the Laptop through USB.
\item There is a button and an LED to the left of the USB port on the Vaman.There is another button to the right of the LED.
\item Press the right button first and immediately press the left button.The LED will be blinking green.The Vaman is now in bootloader mode.
\end{enumerate}
\subsection{The steps for implementation:}
\begin{enumerate}
\item Login to termux-ubuntu on the android device and execute the following commands:\\
Make sure that the required installation of pygmy-sdk had done prior executing below commands
\begin{lstlisting}
proot-distro login debian
cd  /data/data/com.termux/files/home/
mkdir arm
svn co https://github.com/Prathyushakorepu/FWC/blob/main/arm/codes/src/main.c
\end{lstlisting}
\begin{lstlisting}
cd codes/GCC_Project
make
scp /data/data/com.termux/files/home/arm/codes/GCC_Project/output/bin/codes.bin usernameofpc@IPaddress:/home/username
\end{lstlisting}
Make sure that the appropriate username,IP address of the Laptop is given in the above command.
\item Now execute the following commands on the Laptop terminal\\
Make sure that required installation of programmer application and modification of bash file had done prior executing below command
\begin{lstlisting}
bash flash.sh codes.bin
\end{lstlisting}
\item After finishing the process of flashing with the programmer application press the button to the right of the USB port to reset. Vaman is now flashed with our source code
\end{enumerate}
\end{document}
