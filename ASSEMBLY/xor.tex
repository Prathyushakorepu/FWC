\documentclass{article}
\usepackage{circuitikz}
\usepackage{hyperref}
\usepackage[utf8]{inputenc}
\usepackage{graphics}
\title{ASSIGNMENT-1  

Designing XOR with NOR Gates}
\author{PRATHYUSHA KOREPU}
\date{August 2022}

\begin{document}

\maketitle

\section{Introduction}

1)Abstract—This manual shows how to  implement XOR using NOR Gates through Arduino.
\vspace{5MM}

\textbf{Connections :-}
\begin{tabular}{|c|c|c|}

\hline
\textbf{Arduino} & {2} & {GND} \\
\hline
\textbf{LED} & {+ve} & {-ve} \\
\hline
%\textbf{Bread board} & { } & {Q} \\
\end{tabular}

\vspace{5MM}
1.1. {XOR GATE :-}
The output is HIGH(1) if only if one of the inputs is HIGH.If both the inputs are LOW or HIGH, then the output is LOW(0).When the two inputs are different it produce HIGH value.
\vspace{5MM}

\textbf{TRUTH TABLE :-}
\vspace{5MM}
\newline

\begin{tabular}{|c|c|c|}
\hline
\textbf{A} & {B} & {Q} \\
\hline
0 & 0 & 0 \\
\hline
0 & 1 & 1 \\
\hline
1 & 0 & 1 \\
\hline
1 & 1 & 0 \\
\hline


\end{tabular}
\vspace{3mm}
\\
\centering{\textbf{Table-1}}
\vspace{5MM}
\newline

\textbf{BOOLEAN EXPRESSION :-}
\vspace{5MM}
\newline
\begin{equation}
    Q = AB' + BA'
\end{equation}
\textbf{2. XOR with NOR Gates :-}


    $Q = AB' + BA'$

    $Q = A'.B + A.B' + A.A'+ B.B'$  

    $Q = A'(A + B) + B'(A + B)$

    $Q = (A + B) (A'+ B')$ \hspace{8cm} (2)

Take complement on both sides to equation (2)

$Q' = ((A + B)(A' + B'))'$

$Q' = (A + B)' + (A' + B')'$ \hspace{2cm} (Demorgan's theorem)  \hspace{2cm}    (3)

Take complement on both sides to equation (3)

$Q = ((A + B)' + (A' + B')')'$


\textbf{CIRCUIT DIAGRAM}
\\
\vspace{3mm}



\begin{circuitikz} \draw

(0,2) node[nor port] (nor1) {}
(nor1.out) node(x0) [anchor=south west] {$x0=\overline{A}$}
(nor1.in 1) -- ++ (-2,0) node[circ]{} node[left]{$A$}

(1,0) node[nor port] (nor2) {}
(nor2.in 1) -- ++ (0,0) node[circ]{} node[left]{$B$}
(nor2.out) node(x1) [anchor=north west] {$x1=\overline{B}$}
(3,1) node[nor port] (nor3) {}
(nor3.out) node(x2) [anchor=south west] {$x2=\overline{x0+x1}$}
(3,-1) node[nor port] (nor4) {}
%(nor4.in 2) -- ++ (-4,0) node[circ]{} node[left]{$A$}
%(nor4.in 1) -- ++ (-2,0) node[circ]{} node[left]{$B$}
(nor4.out) node(x3) [anchor=north west] {$x3=\overline{A+B}$}
(5,0) node [nor port] (nor5) {}
(nor5.out) node(Q) [anchor=north west] {$Q=\overline{x2+x3}$}
(nor1.out) -| (nor3.in 1)
(nor2.out) -| (nor3.in 2)
(nor3.out) -| (nor5.in 1)
(nor4.out) -| (nor5.in 2)
(nor1.in 1) |- (nor4.in 2)
(nor2.in 1) |- (nor4.in 1);

\end{circuitikz}
\vspace{5MM}
\newline
\textbf{Fig-1}
\vspace{5MM}

Download the following code using the arduino IDE and execute

\href{https://github.com/Prathyushakorepu/FWC}{https://github.com/Prathyushakorepu/FWC}




\end{document}
